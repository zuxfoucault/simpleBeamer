\documentclass{beamer}
\mode<presentation>
{
 \usetheme{Singapore}
% \setbeamercoverd{transparent}
}
% \let\Tiny=\tiny

\usepackage[american]{babel}
\usepackage{csquotes}
\usepackage[sorting=nyt,style=apa]{biblatex} %use biblatex package
\DeclareLanguageMapping{american}{american-apa}
\addbibresource{./test.bib}                  % where is .bib

\begin{document}

	\begin{frame}
		\frametitle{exam.}
		something,
		\begin{definition}
		     A \alert{prime number} is a number that has exactly two divisors.
	        \end{definition}
		\begin{itemize}
		\item 2 is prime (two divisors: 1 and 2).
			\pause
	        \item 3 is prime (two divisors: 1 and 3).
		\end{itemize}

		\begin{enumerate}
				\item<1-> Suppose $p$ were the largest prime number.
				\item<2-> Let $q$ be the product of the first $p$ numbers.
				\item<3-> Then $q + 1$ is not divisible by any of them.
			        \item<1-> But $q + 1$ is greater than $1$, thus divisible by some prime 
					number not in the first $p$ numbers.\qedhere
		\end{enumerate}

	\end{frame}

	\begin{frame}
		\frametitle{example}
		\begin{itemize}
		\item Attributional similarity
		\bigskip
		\item Relational similarity
			\cite{Lin2009} % here is the cite
		\end{itemize}
		\begin{columns}[t]
			    \begin{column}{0.49\textwidth}
				          \begin{block}{This is a headline}
						          \begin{quote}
								Here is some text that is shrunk a little bit isn't it?
							  \end{quote}
				          \end{block}
	      		     \end{column}			
			     \begin{column}{0.49\textwidth}
		             \begin{block}{}
			      This is another block
			     \end{block}
			     \end{column}
		\end{columns}
	        \end{frame}

		\begin{frame}
			\printbibliography % here is the bibliography
		\end{frame}
        
\end{document}

